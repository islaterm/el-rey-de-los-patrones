\begin{Answer}[ref=ex:composite-1]
  Para resolver este ejercicio utilizaremos \textit{composite pattern}, este patrón se 
  puede utilizar en cualquier estructura que deba recorrerse de forma recursiva y que 
  pueda modelarse como un árbol.
  
  En este caso, la estructura de cartas que tenemos es un árbol, donde cada carta puede 
  referenciar a 0 o más cartas, para aplicar este patrón se necesita crear una interfaz 
  común para los objetos que foman parte de la estructura.
  En nuestro caso esa interfaz ya la hemos definido en \java{ICard}, pero habrá que 
  agregarle un método que pueda recorrer el árbol, así:

  \begin{minted}[autogobble, bgcolor=LightGray]{java}
    // com.github.islaterm.yugi.model.card.ICard
    void applyEquippedEffects();
    // com.github.islaterm.yugi.model.card.AbstractCard
    protected Set<ICard> equippedCards = new HashSet<>();

    @Override
    public void applyEquippedEffects() {
      for (ICard card : equippedCards) {
        card.useEffect();
        card.applyEquippedEffects();
      }
    }
  \end{minted}

  De esta forma los efectos se aplicarán recursivamente y terminarán una vez que se llegue
  a una carta que no tenga ninguna otra carta equipada. 
\end{Answer}
%