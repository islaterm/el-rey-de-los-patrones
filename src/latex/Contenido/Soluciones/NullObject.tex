\begin{Answer}[ref={ex:null-1}]
  Para resolver este problema debemos crear una clase que represente los efectos de los
  monstruos y hechizos.
  
  Adicionalmente, necesitamos una forma de definir el caso en que un monstruo no tenga 
  efecto, para esto último utilizaremos \textit{null object pattern}.
  
  Primero, necesitaremos una interfaz para los efectos y permitir que las cartas puedan
  tener dichos efectos:
  \begin{minted}[autogobble, bgcolor=LightGray]{java}
    // com.github.islaterm.yugi.effect.IEffect
    public interface IEffect {
      void use();
    }
    // com.github.islaterm.yugi.card.ICard
    void useEffect();
    // com.github.islaterm.yugi.card.AbstractCard
    @Override
    public void useEffect() {
      cardEffect.use();
    }      
  \end{minted}

  Ahora, podemos implementar los efectos concretos:
  \begin{minted}[autogobble, bgcolor=LightGray]{java}
    // com.github.islaterm.yugi.effect.AbstractEffect
    public class Effect implements IEffect {
      @Override
      public void use() {
        System.out.println("Soy un efecto owo");
      }
    }
    // com.github.islaterm.yugi.effect.NullEffect  
    public final class NullEffect implements IEffect {
      @Override
      public void use() {
        System.out.println("No hago nada :D");
      }
    }
  \end{minted}
  La clase \java{NullEffect} se define como \java{final}, esto significa que la clase no
  puede ser extendida.
  Esto ayuda a entender mejor el propósito de la clase.
  
  Con los efectos definidos podemos crear los constructores respectivos de las cartas 
  para definir sus efectos.
  \begin{minted}[autogobble, bgcolor=LightGray]{java}
    // com.github.islaterm.yugi.card.monster.BasicMonsterCard
    public BasicMonsterCard(IEffect effect) {
      cardEffect = effect;
    }
    // Los otros tipos de monstruos son análogos
    // com.github.islaterm.yugi.card.SpellCard
    public SpellCard(Effect effect) {
      cardEffect = effect;
    }
  \end{minted}

  Note que el constructor de las cartas de hechizos no puede recibir efectos nulos 
  porque no tendría sentido.
\end{Answer}