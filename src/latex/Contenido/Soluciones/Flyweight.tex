\begin{Answer}[ref={ex:flyweight-1}]
  Para evitar crear un mismo objeto dos veces ya se vio que se puede utilizar un 
  \textit{singleton}, pero si se tienen muchos objetos que se requiere que sean únicos
  entonces implementar este patrón agregaría mucha complejidad al código.
  
  En vez de eso, se puede utilizar \textit{factory pattern} para unificar y simplificar la
  creación de dichos objetos.
  Ahora, si además de eso el \textit{factory} guarda los objetos que ha creado para no 
  crearlos de nuevo, estaremos utilizando \textit{flyweight factory pattern}.

  \begin{minted}[autogobble, bgcolor=LightGray]{java}
    // com.github.islaterm.yugi.model.effect.FlyweightEffectFactory
    public class FlyweightEffectFactory {
      private Map<String, IEffect> createdEffects = new HashMap<>();

      public IEffect getEffect(String effectId) {
        if(!createdEffects.containsKey(effectId)) {
          createdEffects.put(effectId, new Effect());
        }
        return createdEffects.get(effectId);
      }
    }
  \end{minted}
\end{Answer}