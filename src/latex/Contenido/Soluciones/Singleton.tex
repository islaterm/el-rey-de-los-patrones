\begin{Answer}[ref={ex:singleton-1}]
  Para asegurarse de que no se pueda crear más de una instancia de un objeto se necesita
  ocultar el constructor de la clase.
  Esto implica que la clase sólo puede ser instanciada desde dentro de ella, por lo que es
  necesario que el método para obtener la instancia de la clase sea estático.
  Esto es lo que se conoce como \textit{singleton pattern}.

  Modifiquemos la clase \java{NullEffect} para que sea un \textit{singleton}.
  \begin{minted}[autogobble, bgcolor=LightGray]{java}
    // com.github.islaterm.yugi.effect.NullEffect
    public class NullEffect implements IEffect {
      private static IEffect instance;

      private NullEffect() {
      }

      @Override
      public void use() {
        System.out.println("No hago nada :D");
      }

      public static IEffect getInstance() {
        if (instance == null) {
          instance = new NullEffect();
        }
        return instance;
      }
    }
  \end{minted}
\end{Answer}