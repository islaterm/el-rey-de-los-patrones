\begin{Answer}[ref={ex:template-1}]
    Lo primero que necesitamos hacer es crear una interfaz y una clase abstracta que 
    represente todos los tipos de cartas, así:
    \begin{minted}[autogobble, bgcolor=LightGray]{java}
      // com.github.islaterm.yugi.model.card.ICard
      public interface ICard {
        void playToMat(GameMat gameMat);
      }

      // com.github.islaterm.yugi.model.card.AbstractCard
      public abstract class AbstractCard implements ICard {
        @Override
        public abstract void playToMat(GameMat gameMat);
      }
    \end{minted}

    Lo que estamos haciendo al definir el método como abstracto es relegar la labor de 
    decidir cómo se jugarán las cartas a las implementaciones particulares de cada tipo de
    carta.
    Esto es lo que se conoce como \textit{template method pattern}.

    Ahora, las implementaciones particulares de cada método serán:
    \begin{minted}[autogobble, bgcolor=LightGray]{java}
      // com.github.islaterm.yugi.model.card.monster.BasicMonsterCard
      public class MonsterCard extends AbstractCard {

        @Override
        public void playToMat(@NotNull GameMat gameMat) {
          gameMat.addMonster(this);
        }
      }

      // com.github.islaterm.yugi.model.card.SpellCard
      public class SpellCard extends AbstractCard {
        @Override
        public void playToMat(@NotNull GameMat gameMat) {
          gameMat.addSpell(this);
        }
      }
    \end{minted}
  \end{Answer}
  
  \begin{Answer}[ref={ex:template-2}]
    Este problema también puede ser resuelto utilizando un \textit{template}.
    Para esto necesitaremos extender el modelo de cartas de monstruos que llevamos hasta 
    ahora.
    \begin{minted}[autogobble, bgcolor=LightGray]{java}
      // com.github.islaterm.yugi.model.card.monster.IMonsterCard
      public interface IMonsterCard extends ICard {
        void addTribute(IMonsterCard card);
      }
    \end{minted}

    \begin{minted}[autogobble, bgcolor=LightGray]{java}
      // com.github.islaterm.yugi.model.card.monster.AbstractMonsterCard
      public abstract class AbstractMonsterCard extends AbstractCard implements IMonsterCard {
        private Set<IMonsterCard> tributes = new HashSet<>();

        @Override
        public void addTribute(IMonsterCard card) {
          this.tributes.add(card);
        }
      
        @Override
        public void playToMat(GameMat gameMat) {
          if (enoughSacrifices()) {
            gameMat.addMonster(this);
            tributes.forEach(gameMat::removeMonster);
          }
        }

        protected abstract boolean enoughSacrifices();
      }
    \end{minted}

    Ahora, con esas clases definidas se pueden implementar los tres tipos de cartas de 
    monstruos que necesitamos.
    \begin{minted}[autogobble, bgcolor=LightGray]{java}
      // com.github.islaterm.yugi.model.card.monster.BasicMonsterCard
      public class BasicMonsterCard extends AbstractMonsterCard {
        @Override
        protected boolean enoughSacrifices() {
          return tributes.isEmpty();
        }
      }
      // com.github.islaterm.yugi.model.card.monster.OneTributeMonsterCard
      public class OneTributeMonsterCard extends AbstractMonsterCard {
        @Override
        protected boolean enoughSacrifices() {
          return tributes.size() == 1;
        }
      }
      // com.github.islaterm.yugi.model.card.monster.TwoTributesMonsterCard
      public class TwoTributesMonsterCard extends AbstractMonsterCard {
        @Override
        protected boolean enoughSacrifices() {
          return tributes.size() == 2;
        }
      }
    \end{minted}
  \end{Answer}