\section{Definiciones necesarias}
  Sea el juego de cartas \textit{Yu-Gi-Oh!} definido como la tupla 
  $Y = $\dots\\
  Sería interesante (?) dar una definición matemática para el juego, pero no lo voy a 
  hacer.

  \textit{Yu-Gi-Oh!} es un \textit{TCG} con varios tipos de cartas, pero para los 
  siguientes ejercicios nos centraremos en 2 tipos en particular, \textit{cartas de 
  monstruos} y \textit{cartas de hechizos}.

\section{Template method}
  \begin{Exercise}[title={Zonas de juego}, label={ex1}]
    Si bien existen muchas diferencias entre estos dos tipos de cartas, por ahora 
    consideraremos que solo se diferencian en la zona del campo en la que son jugadas.

    Defina las cartas de monstruos y de hechizos para que puedan jugarse en su zona 
    correspondiente siguiendo buenas metodologías de diseño.
  \end{Exercise}

  \begin{Answer}[ref={ex1}]
    Lo primero que necesitamos hacer es crear una interfaz y una clase abstracta que 
    represente todos los tipos de cartas, así:
    \begin{minted}[autogobble, bgcolor=LightGray]{java}
      // com.github.islaterm.yugi.card.ICard
      public interface ICard {
        void playtoMath(GameMat gameMat);
      }

      // com.github.islaterm.yugi.card.AbstractCard
      public abstract class AbstractCard implements ICard {
        @Override
        public abstract void play();
      }
    \end{minted}

    Lo que estamos haciendo al definir el método como abstracto es relegar la labor de 
    decidir cómo se jugarán las cartas a las implementaciones particulares de cada tipo de
    carta.
    Esto es lo que se conoce como \textit{template method pattern}.

    Ahora, las implementaciones particulares de cada método serán:
    \begin{minted}[autogobble, bgcolor=LightGray]{java}
      // com.github.islaterm.yugi.card.MonsterCard
      public class MonsterCard extends AbstractCard {

        @Override
        public void playtoMath(@NotNull GameMat gameMat) {
          gameMat.addMonster(this);
        }
      }

      // com.github.islaterm.yugi.card.SpellCard
      public class SpellCard extends AbstractCard {
        @Override
        public void playtoMath(@NotNull GameMat gameMat) {
          gameMat.addSpell(this);
        }
      }
    \end{minted}
  \end{Answer}
  %
%  