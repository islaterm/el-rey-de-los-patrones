\section{Template method}
  \begin{Exercise}[title={Zonas de juego}, label={ex:template-1}, difficulty=1]
    Si bien existen muchas diferencias entre estos dos tipos de cartas, por ahora 
    consideraremos que solo se diferencian en la zona del campo en la que son jugadas.

    Defina las cartas de monstruos y de hechizos para que puedan jugarse en su zona 
    correspondiente siguiendo buenas metodologías de diseño.
  \end{Exercise}

  \begin{Exercise}[
      title={Niveles de monstruos}, 
      label={ex:template-2}, 
      difficulty=1
    ]
    Además de los tipos de cartas del ejercicio \ref{ex:template-1} se tienen distintos 
    niveles de monstruos que requerirán sacrificios para ser jugados.
    Los niveles son:
    \begin{itemize}
      \item \textbf{Niveles 1 a 4:} Se pueden jugar sin necesidad de hacer sacrificios.
      \item \textbf{Niveles 5 y 6:} Necesitan de un sacrificio para ser jugados.
      \item \textbf{Niveles 7 y superiores:} Necesitan de 2 sacrificios para jugarse.  
    \end{itemize}

    Modifique el código para permitir esta mecánica.
  \end{Exercise}

  \begin{Exercise}[
      title={Invocaciones de monstruos (Propuesto)}, 
      label={ex:template-3}, 
      difficulty=2
    ]
    Extienda el código anterior para permitir dos tipos de invocaciones de monstruos, en
    modo de defensa boca abajo y boca arriba en modo de ataque.
  \end{Exercise}
% 