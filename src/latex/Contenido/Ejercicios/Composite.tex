\section{Composite}
  \begin{Exercise}[
      title={Cartas de efecto continuo},
      label={ex:composite-1}, 
      difficulty=2
    ]
    Considere ahora que existen efectos que afectan a otras cartas y que se mantienen a lo
    largo del juego, en este caso se definen 2:
    \begin{itemize}
      \item \textbf{Cartas de efecto continuo:} Son cartas que tienen un efecto sobre una 
        o más cartas en juego.
      \item \textbf{Cartas de equipo:} Son cartas que pueden ser equipadas por un monstruo 
        y que tienen un efecto sobre este.
    \end{itemize}

    Como estas cartas tienen efectos sobre otras y se necesita saber qué cartas son las 
    que las están afectando, se necesita guardar referencias a estas.
    Esto puede causar que haya cartas con referencias a otras que a su vez tienen 
    referencias a otras.
    Implemente esta nueva funcionalidad de las cartas. 
  \end{Exercise}

  \begin{Exercise}[title={Cartas de campo}, label={ex:composite-2}, difficulty=4]
    Se le solicita ahora que agregue un nuevo tipo de carta al juego.

    Las cartas de campo son cartas que se juegan en una zona propia del \textit{game mat}
    y que tienen efecto sobre todo el resto de las cartas.

    Extienda el código para agregar esta mecánica:

    \textbf{¡CUIDADO!} Si se implementa de la misma manera que en el ejercicio anterior se
    puede caer en referencias circulares (por ejemplo si hay 2 cartas de campo en juego).
  \end{Exercise}
%