\section{Null Object}
  \begin{Exercise}[title={Efectos}, label={ex:null-1}, difficulty=1]
    Además de los tipos ya definidos, todas las cartas de hechizos tienen efectos, y las 
    cartas de monstruos pueden o no tener algún efecto.
    Extienda la implementación para permitir esta mecánica.
  \end{Exercise}

  \begin{Exercise}[
      title={Efectos específicos (Propuesto)}, 
      label={ex:null-2}, 
      difficulty=3
    ]
    Actualmente la implementación tiene efectos pero estos no hacen mucho.
    Considere los siguientes efectos específicos:
    \begin{itemize}
      \item Utilizar el efecto de una carta puede anular el efecto de otra.
      \item El efecto de una carta puede ocupar espacios de la zona de monstruos o 
        hechizos con \textit{tokens} que no pueden ser movidos o sacrificados y que no 
        pueden realizar ninguna acción.
    \end{itemize}

    Extienda el código para implementar estas funcionalidades.
  \end{Exercise}
%